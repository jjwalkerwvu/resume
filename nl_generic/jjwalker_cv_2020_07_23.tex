%%%%%%%%%%%%%%%%%%%%%%%%%%%%%%%%%%%%%%%%%
% "ModernCV" CV and Cover Letter
% LaTeX Template
% Version 1.1 (9/12/12)
%
% This template has been downloaded from:
% http://www.LaTeXTemplates.com
%
% Original author:
% Xavier Danaux (xdanaux@gmail.com)
%
% License:
% CC BY-NC-SA 3.0 (http://creativecommons.org/licenses/by-nc-sa/3.0/)
%
% Important note:
% This template requires the moderncv.cls and .sty files to be in the same 
% directory as this .tex file. These files provide the resume style and themes 
% used for structuring the document.
%
%%%%%%%%%%%%%%%%%%%%%%%%%%%%%%%%%%%%%%%%%

%----------------------------------------------------------------------------------------
%	PACKAGES AND OTHER DOCUMENT CONFIGURATIONS
%----------------------------------------------------------------------------------------

\documentclass[12pt,a4paper,sans]{moderncv} % Font sizes: 10, 11, or 12; paper sizes: a4paper, letterpaper, a5paper, legalpaper, executivepaper or landscape; font families: sans or roman

\moderncvstyle{classic} % CV theme - options include: 'casual' (default), 'classic', 'oldstyle' and 'banking'
\moderncvcolor{blue} % CV color - options include: 'blue' (default), 'orange', 'green', 'red', 'purple', 'grey' and 'black'

\usepackage{lipsum} % Used for inserting dummy 'Lorem ipsum' text into the template

\usepackage[scale=0.75]{geometry} % Reduce document margins
%\setlength{\hintscolumnwidth}{3cm} % Uncomment to change the width of the dates column
%\setlength{\makecvtitlenamewidth}{10cm} % For the 'classic' style, uncomment to adjust the width of the space allocated to your name
% use pdfpages to include additional pdf files!!!
\usepackage[final]{pdfpages}	

%% can't get hyperlinks to work, due to an option clash for package hyperref 
%\usepackage{hyperref}
%\hypersetup{colorlinks=true,linkcolor=blue}


%----------------------------------------------------------------------------------------
%	NAME AND CONTACT INFORMATION SECTION
%----------------------------------------------------------------------------------------

\firstname{Jeffrey} % Your first name
\familyname{Walker} % Your last name

% All information in this block is optional, comment out any lines you don't need
\title{Curriculum Vitae}
%\title{List of Publications}
\address{IJburglaan 1055}{1087 EN Amsterdam, Netherlands}%{Netherlands}
%\address{2485 Bauer Rd}{Eden, NY 14057, USA}
%\mobile{+1 (716) 725 2729}
\mobile{+31 (0)64 373 7950}
%\phone{+1 (319) 353 2320}
%\fax{(000) 111 1113}
\email{jjwalkerwvu@gmail.com}
%\homepage{www.linkedin.com/in/jeffrey-walker-20661079}{www.linkedin.com/in/jeffrey-walker-20661079} % The first argument is the url for the clickable link, the second argument is the url displayed in the template - this allows special characters to be displayed such as the tilde in this example
\extrainfo{Citizen of The United States\\Authorized to work in The Netherlands}
\photo[70pt][0.4pt]{jjw_picture.jpg} % The first bracket is the picture height, the second is the thickness of the frame around the picture (0pt for no frame)
%\quote{"A quotation if you want." - Some Guy}

\begin{document}

%----------------------------------------------------------------------------------------
%	COVER LETTER
%----------------------------------------------------------------------------------------


%% To remove the cover letter, comment out this entire block.

%% Note: %!% indicates lines of text I used when composing my cover letter for ASH last time, in August 2018

%!%\clearpage

%\recipient{recipient name}{Department of Statistics\\414 Van Allen Hall \\University of Iowa %}
%!%\recipient{Hiring Manager}{American School of The Hague\\The Hauge} %% Letter recipient
%% Put email address in or not?
%e.camporeale@cwi.
%!%\date{\today} % Letter date
%!%\opening{To whom it may concern,} % Opening greeting




%!%\closing{Thank you for your consideration,} %% Closing phrase
%% Do I need the line below?
%\enclosure[Attached]{curriculum vit\ae{}} % List of enclosed documents enclosed

%% Print letter title
%!%\makelettertitle 

%%%%%
%% Opening
%% Indicate here what function you are applying for and mention how you came to know about the vacancy, or make a light-hearted opening with which you can immediately make an impression, surprise the reader, for that distinguishes you from others
%% mention the position, e.g., pos. ref. 432
%I am a former space scientist with a background in analyzing time series spacecraft data and in developing numerical methods to answer challenging problems. 
%!%I am a former teaching assistant for a large state university in the United States. 
%!%I have nearly ten years of teaching experience in a lecture and laboratory setting, and have taught hundreds of students.
%!%I have a strong foundation in mathematics, science, and programming, and I am comfortable teaching as a substitute in any of these subjects.
%I have strong simulation and analysis skills that I hope I can bring to the role as a Software Engineer for Trinamics.
%I am looking to transition to Quantitative Analysis in the finance industry.
% Might want something like: I am looking to use the skills I have developed in order to tackle business problems.
% say some words as to how I hope to help with: working closely with different business units to improving and redesign our Smarter Lending, Customer Engagement and Customer Convenience capabilities as to bring ING Group to the forefront of analytic excellence
%I am eager to apply my analytical skills and statistical knowledge into producing quantitative, actionable insight for Ahold Delhaize in the role of Data Scientist.
%I seek to leverage my experience with spacecraft analysis to support mission modelling in the role of Modelling and Simulation Engineer with RHEA group.
%I am eager to apply my analytical skills and statistical knowledge into producing quantitative, actionable insight for IQVIA in the role as Junior Statistician Analyst (R1026596).
%I am eager to leverage my applied mathematics and stastics background at Kyos in the role as a Quantitative Developer.
%!%\vspace{0.5cm}

%% Motivation
%% Indicate why you aspire this function and why you chose for this organisation/company? 
%% Perhaps mention or describe what led you to apply. 
%% Whenever you send an open application, it's extremely important that you make it clear what function(s) you're interested in
%%
%!%In my previous work as a teaching assistant, I planned and presented structured material to demonstrate and clarify physics concepts in ways relatable to their own experience.
%I worked with first-year college students to demonstrate and clarify physics concepts in ways relatable to their own experience.
%!%I also helped demystify mathematics concepts that had gone unaddressed from prerequisite courses. 
%In my previous work as a scientist, I applied fundamental mathematics and physics concepts to answer research relevant questions.
%!%I enjoy teaching and I think my experience working within the American education system at a large state university in the United States, while studying mathematics and physics at a deep level, provides me with a good perspective on how to prepare students for higher education, and how to inspire a passion for learning math and science. 
%In my previous work as a physicist, I worked to develop numerical models, and simulated physical phenomena to provide real-world predictions. 
%In my previous work as a physicist, I have made tailor-made analytical and numerical solutions to answer real-world problems.  
%I have performed statistical analysis on spacecraft data to find, explain, and quantify unexpected features in these data. 
%My time working spacecraft data has interested me in the problems in planning and supporting spacecraft missions.
%I have experience analyzing inherently stochastic phenomena, such as in my work in the form of charge fluctuations on particulate matter in plasma, and in determining casual relationships from different spacecraft data.
%In addition to my foundation in statistical analysis, I understand how to apply fundamental machine learning techniques, such as support vector machines, recommender systems, k-means clustering, online learning, and neural networks.
%I am interested in how to apply my current skillset to optimize trading strategies.
% Do I need to mention my simulation ability?
%I have experience with object oriented programming, in the use and development of simulation codes.
%I have extensive experience developing and running simulation codes to solve physics problems.
%!%\vspace{0.5cm}
%Developing predictive and prescriptive models with large and varied dataset
%Professional and/or academic experience in Big Data analytics & deployment of models and algorithms to solve real-world problems (with deep statistical modelling expertise)

%% Suitability
%% Why are you suitable for this position? Which of your competencies fits the requirements described in the advertisement?
%%
%I have experience with lower level languages C$^{++}$, Fortran, higher level languages like Python, and ten years of experience with Matlab. 
%I have experience running specialized simulation codes in Fortran, as well as a solid understanding of C$^{++}$. 
%I also have strong skills in data analysis using Matlab and Python.
%!%From my scientific computational work performed at state universities in the United States, I also have first-hand experience understanding the practical skills needed to be successful in higher learning or in the workplace. 
%!%Computational knowledge and skills are of ever-increasing importance in our global society, and I think that I can be helpful to students who have aspirations to learn programming.
%I have studied mathematics and physics at a deep level, 
%I am capable of presenting to a non-technical audience in clear and accessible language.
%I am always thinking of different ways to interpret fundamental ideas, which I think is helpful to teaching students from different backgrounds and outlooks.
%I also know how to use SQL to query databases.
%I have not used SysMl, but I am motivated to learn in order to perform simulations of space missions.
%I have experience with Python, particularly  NumPy, SciPy, and matplotlib packages.
%As evidenced by my publications, I am capable of producing graphical visualizations to tell a story about the (underlying data). 
%I am excited by the prospect of working on big data sets to solve real-world problems.
%I am able to write texts on quantitative topics in a clear and accessible language, and explain my results to colleagues from different disciplines.
%I enjoy the challenge of understanding mathematical models, questioning their assumptions, and checking the correctness of their implementation.
%In addition to my motivation to strive for the best quality in my own work, I like being part of a team and contributing to the group's efforts.
%I thrive on the opportunity to learn new skills and face fresh challenges.
%I enjoy the challenge of translating business challenges into analytical problems, and finding data driven solutions
%to understand all aspects of a proposed model and check the correctness of its implementation against our own code.
% I am onboard with the approach of questioning the mathematical models, (looking beyond tasks and claims), understanding all aspects of a proposed challenging mathematical assumptions behind models, thoroughly vetting (models) before they are delivered for trading purposes. 
%!%\vspace{0.5cm}

%% Closing: 
%% Make sure to give a powerful closing and ask for an interview. Refer to the enclosed CV
%Given my analytical background, and the initiative that I have taken to study the machine learning techniques in my own personal time, I am confident that I can succeed as a Quantitative Developer with Kyos. 
%Given my analytical background, and the initiative that I have taken to study financial markets in my own personal time, I am confident that I can succeed as a Software Engineer with Trinamics.
%!%Given my passion for mathematics, physics, programming, and teaching, I am confident that I can be helpful to the American School of the Hague as a substitute teacher.
%I look forward to your response.




%%
% Here is the same letter, but adapted to indeed.com format:
% (Just uncomment and paste!)
%%

%To whom it may concern,

%I am a former space scientist with a background in analyzing time series spacecraft data and in developing numerical methods to answer challenging problems. I am eager to apply my analytical skills and statistical knowledge into producing quantitative, actionable insight for Takeaway.com in the role as a forecasting analyst.

%In my previous work as a plasma physicist, I worked to develop numerical models, and simulate physical phenomena to provide real-world predictions. I worked with spacecraft data to find, explain, and quantify unexpected features in these data.  In addition to my foundation in statistical analysis, I understand how to apply fundamental machine learning techniques, such as support vector machines, recommender systems, k-means clustering, online learning, and neural networks.

%I have experience with lower level languages C++, Fortran and higher level langauges such as Matlab and Python, including NumPy, SciPy, and matplotlib packages. I am able to write texts on quantitative topics in a clear and accessible language, and explain my results to colleagues from different disciplines. I enjoy the challenge of understanding mathematical models, questioning their assumptions, and checking the correctness of their implementation. In addition to my motivation to strive for the best quality in my own work, I like being part of a team and contributing to the group's efforts. I thrive on the opportunity to learn new skills and face fresh challenges.

%Given my analytical background, and the initiative that I have taken to study machine learning techniques in my own personal time, I am confident that I can succeed in the forecasting analyst role for Takeaway.com. I look forward to your response.

%Thank you for your consideration,

%Jeffrey J. Walker

%!%\makeletterclosing %% Print letter signature

%!%\clearpage

%----------------------------------------------------------------------------------------
%	After cover letter, begin with the CV
%----------------------------------------------------------------------------------------
\makecvtitle % Print the CV title

%----------------------------------------------------------------------------------------
%	Personal Profile
%----------------------------------------------------------------------------------------
%% The Projob recruiters recommend putting in a personal profile section, or an objective
%% Keep it to one sentence or less!

%\section{Personal Profile}
% include machine learning with statistical analysis?
% "with (n years) of experience?
%\cvitem{}{I am a computational research scientist looking to apply statistical analysis and develop numerical methods to inform business decisions}{}
%% Here's a different profile to use when making an academic resume?
%\cvitem{}{I am an experienced teacher and computational physicist looking to inspire students in the fields of physics and numerical methods}{}
%\begin{itemize}
%\item
%\item
%\end{itemize}

%----------------------------------------------------------------------------------------
% perhaps place references here:
%----------------------------------------------------------------------------------------
%\section{References}
%\cvitem{}{Jasper Halekas, email: jasper-halekas@uiowa.edu}
%\cvitem{}{Mark Koepke, email: mark.koepke@mail.wvu.edu}
%\cvitem{}{Michael Zimmerman, email: Michael.Zimmerman@jhuapl.edu}

%----------------------------------------------------------------------------------------
%	EDUCATION SECTION
%----------------------------------------------------------------------------------------

\section{Education}


\cventry{August 2008 -- May 2015}{Doctor of Philosophy}{West Virginia University \textcolor{white}{another school here}}{}{}{Concentration in Plasma Physics}
%\cventry{August 2008 -- May 2015}{Doctor of Philosophy}{West Virginia University}{}{}{Concentration in Plasma Physics}
\cventry{August 2005 -- August 2008}{Masters of Science}{West Virginia University}{}{}{Concentration in Plasma Physics}  % Arguments not required can be left empty
%\cventry{2005--2008}{Masters of Science}{West Virginia University}{}{}{Concentration in Plasma Physics}
\cventry{August 2001 -- May 2005}{Bachelor of Science}{Canisius College}{}{}{Major in Physics, Minor in Mathematics}
%% I left the commented line below, in case I would rather use "specialized" instead of "concentration in"
%%{Specialized in Physics}

%% Consider removing dissertation section for a regular resume?
%% Use this section to discuss dissertation
%\section{Dissertation}
%% use \cvitem command and fill in the fields
%\cvitem{Title}{\emph{Fine-Particle Charging-Rate-Limit Modification to Grain Dynamics in Abrupt and Gradual Inhomogeneities}}
%\cvitem{Supervisor}{Professor Mark Koepke}
%% For an industry job, might want to tailor the description of the dissertation to something with a more of a business angle/perspective.
%\cvitem{Description}{My theoretical and computational work with fine-particle charging shows how non-stationary charging affects grain trajectories in plasma. 
%Dust grains in plasma are commercially important, because their presence can damage computer chips during the manufacturing process, directly affecting the bottom line.
%One of the goals of this project was to assess whether non-stationary charging effects on grain trajectories might be exploited to remove nano-scale dust from plasma-assisted etching reactors.  
%In this project, I assessed how the non-stationary charging effect on grain trajectories might be exploited to remove nano-scale dust from plasma-assisted etching reactors. 
%I developed a symplectic, single-particle code capable of treating the time-dependent charge on a dust grain, to find the effect of charging rate on grain trajectory.
%In order to assess gyro-phase drift in the Magnetized Dusty Plasma Experiment (MDPX) at the University of Auburn, the effects of neutral and ion drag must be modelled. 
%I developed an iterative numerical scheme to treat these non-linear drag terms in the velocity update term. 
%Predictions were made for grain trajectories in the Magnetized Dusty Plasma Experiment (MDPX) at the University of Auburn.
% Make some note about how mdpx is low temperature, rf plasma?
%%
%}

%----------------------------------------------------------------------------------------
%	WORK EXPERIENCE SECTION
%----------------------------------------------------------------------------------------

% include a section on research speciallity? 
%\section{Research Speciality}

\section{Work Experience}

\cventry{September 2018 -- August 2019}{Data Scientist}{Jugaad Analytics}{Amsterdam, North Holland, Netherlands}{}{
%At Jugaad Analytics, I developed data analytics solutions for clients using a hybrid machine learning and engineering approach 
%At Jugaad Analytics, I developed predictive maintenance solutions for clients using a hybrid machine learning and engineering approach 
Developed predictive maintenance solutions for clients using a hybrid machine learning and engineering approach
\begin{itemize}
%\item Built custom software to predict Dredge pump efficiency for Royal IHC
%\item Successfully delivered custom engineering principles and machine learning model for IHC
\item Analysed raw data sets from Royal IHC to determine how to find value in their data
% Where do I mention 3-4 pumps? or 8-10?
\item Built and deployed custom, scalable database integration and machine learning algorithm in \textsc{Python} to forecast dredge pump efficiency for Royal IHC
\item Forecasted dredge pump efficiency with a mean absolute percent error less than 5\% between prediction and out-of-sample data
%\item Actively collaborated with my stakeholders at Royal IHC to deliver the desired product  
\end{itemize}
}

%%\subsection{Vocational}
\cventry{June 2015 -- June 2017}{Post-Doctoral Researcher with Jasper Halekas}{University of Iowa}{Iowa City, IA, USA}{}{
%Responsible for statistical analysis of NASA spacecraft measurements and developing numerical models to explain features within these datasets. 
Analysed and presented data from NASA's Lunar Atmosphere and Dust Environment Explorer (\textsc{Ladee}) and \textsc{Artemis} missions.
% I used Matlab, IDL, and spice kernels?   
%During my work with Jasper Halekas, 
\begin{itemize}
%\item used \textsc{Artemis} data of solar wind conditions in order to determine events of interest (corresponding to ldex current enhancements/depletions)
%\item I used highly specialized software (Spice Kernels) to accurately retrieve \textsc{Ladee} mission data, such as spacecraft trajectory, orientation, and field of view of instruments
%% might be good, to make these accomplishments more suited to a general audience, by saying something like: "successfully did [stuff] from analyzing time series data"
%\item Determined from time series data that the source of anomalous integrated current in the Lunar Dust Experiment (\textsc{Ldex}) instrument onboard \textsc{Ladee} is due to Energetic Neutral Atoms (ENAs), even though \textsc{Ldex} was not designed to detect them
%\item I used statistical analysis along with a bespoke numerical model to show that reflected hydrogen atoms from the moon's surface cause the mystery source of current in a NASA spacecraft instrument (LDEX), even though the instrument was not designed to measure these energetic hydrogen atoms
%\item With \textsc{Matlab}, I performed statistical analysis using a tailor-made numerical model to show that reflected hydrogen atoms from the moon's surface cause the mystery source of current in a NASA spacecraft instrument (\textsc{Ldex}), even though the instrument was not designed to measure these energetic hydrogen atoms
%\item I discovered signatures of Lunar Magnetic Anomalies in the \textsc{Ldex} time series data, another unexpected but fortuitous result
%\item I used \textsc{Matlab} to discover signatures of Lunar Magnetic Anomalies in the \textsc{Ldex} time series data, another unexpected but fortuitous result
%\item Built data pipeline and conducted analysis of \textsc{Ldex} data using \textsc{Matlab} 
\item Demonstrated that Energetic Neutral Atoms (ENAs) are the best candidate to explain why the mystery source of current in the Lunar Dust Experiment (\textsc{Ldex}) is highly correlated with solar wind flux, having a correlation coefficient of 0.72 % when ni*vi*scattering function is used
%\item I discovered signatures of Lunar Magnetic Anomalies in the \textsc{Ldex} time series data, another unexpected but fortuitous result
%\item I used \textsc{Matlab} to discover signatures of Lunar Magnetic Anomalies in the \textsc{Ldex} time series data, another unexpected but fortuitous result 
%\item I discovered signatures of Lunar Magnetic Anomalies in the \textsc{Ldex} time series data, another unexpected but fortuitous result
%\item I used \textsc{Matlab} to discover signatures of Lunar Magnetic Anomalies in the \textsc{Ldex} time series data, another unexpected but fortuitous result 
%\item Discovered a 30-40\% reduction in \textsc{Ldex} current when Lunar Magnetic Anomalies are in \textsc{Ldex} field of view, consistent with the reduction of ENA flux from these regions, another unexpected but fortuitous result that supports ENAs as the best candidate to explain the mystery current source
%\item Reduced the reflection ratio for the sightlines from these 3 Lmas from 0.16 to 0.12 (33% drop) when predicting ENA flux, with the resulting time series strongly matching the \textsc{Ldex} current
\item Discovered signatures of 3 Lunar Magnetic Anomalies in \textsc{Ldex} time series data when they are in \textsc{Ldex} field of view, another unexpected but fortuitous result that supports ENAs as the best candidate to explain the mystery current source % known to be in ldex fov 
%\item I analysed how solar wind conditions may affect the ENA flux observed from lunar magnetic anomalies
%\item Used Gradient Descent with Non-Negative Least Squares to find LDEX response function to ENA flux
\end{itemize}
}
%%##!!!!!~~~~~
%% I have commented the above for easy copy and paste when filling out a form
%As a Post-Doctoral Researcher with Jasper Halekas, I was responsible for statistical analysis of NASA spacecraft measurements and developing numerical models to explain features within these datasets. I investigated data from NASA's Lunar Atmosphere and Dust Environment Explorer (Ladee) and Artemis missions.

%I Used Spice Kernels to retrieve Ladee mission data, such as spacecraft trajectory,  orientation, and field of view of instruments.

%I Determined from time series data that the source of anomalous integrated current in the Lunar Dust Experiment (Ldex) instrument onboard Ladee is due to Energetic Neutral Atoms (ENAs), even though Ldex was not designed to detect them.

%I Discovered signatures of Lunar Magnetic Anomalies in the Ldex time series  data, another unexpected but fortuitous result .



% With mark, I produced results for presentation at the low temperature plasma science center
\cventry{May 2007 -- May 2015}{Graduate Research Assistant for Mark Koepke}{West Virginia University}{Morgantown, WV, USA}{}
{
Developed simulation codes and analytical models for dust grains with time-dependent charge in plasma
%During my Ph.D. education, I was responsible for development and analysis of simulation codes and analytical models for dust grains with time-dependent charge in plasma.
%I have a strong theoretical background on surfaces in plasmas and extensive hands-on experience with laboratory equipment.
%I collect, analyse, and present technical data from my own symplectic, single particle simulation code with adaptive time-step to my peers in major conferences and peer-reviewed journals. 
%I use my single-particle code to show that gyro-phase drift of magnetized orbit, microscopic dust grains results in nano-particle transport in plasmas, specifically in the absence of neutral drag force. 
%I use my symplectic single-particle code to show that the charging rate of magnetized orbit, microscopic dust grains modifies grain trajectories in plasmas. 
%I regularly implement increasingly sophisticated charge models within this code to see the resulting effect on particulate transport. 
%I have used Particle-In-Cell codes to simulate the effect of plasmoid cross-section shape on magnetic barriers. 
%I have experience with irregular meshing for Magneto Hydrodynamic simulations using Finite Element Analysis and the \textsc{Aranea} mesh generator written by Richard Marchand (University of Alberta). 
%I have procured and installed high vacuum controls and equipment, including mass flow controllers and interlock circuits. 
%%I have co-authored several papers and posters.
%{
%During my work with Mark Koepke,
\begin{itemize}
%\item I demonstrated how magnetized-orbit dust grain transport depends on charging rate for abrupt and gradual inhomogeneities
%\item I demonstrated how magnetized-orbit dust grain transport depends on charging rate for abrupt and gradual inhomogeneities
%in the presence of planar sheath mechanisms and different charge models
%\item With Michael Zimmerman I developed a symplectic, single particle simulation code in \textsc{Matlab} to study dust grain transport due to non-stationary charging effects, including an innovative iterative leapfrog scheme to treat non-linear drag on dust grains
\item Developed a symplectic, single particle simulation code in \textsc{Matlab} to study dust grain transport due to non-stationary charging effects, including an innovative iterative leapfrog scheme to treat non-linear drag on dust grains
%\item I developed an iterative numerical scheme to treat non-linear drag on dust grains
% In other words, the semi-analytic theory
%\item I developed theory for dust grain transport in an abrupt plasma inhomogeneity. %This model provides an intuitive understanding of how charged dust grains move as a result of the ratio between charging rate and gyro-period
%\item I developed an intuitive analytical theory to describe how dust grains move in an abrupt plasma inhomgeneity, even though this problem appears intractable 
\item Developed an intuitive analytical theory to describe how dust grains move in an abrupt plasma inhomgeneity
\item Predicted grain trajectories for the Auburn Magnetized Dusty Plasma experiment using my simulation and analytical theory % Sets limits of detection!
%\item I demonstrated the effect of plasmoid cross-section on plasmoid penetration of a magnetic barrier using \textsc{Matlab} analysis on \textsc{Fortran} Particle-In-Cell simulations
%\item I demonstrated the effect of plasmoid cross-section on plasmoid penetration of a magnetic barrier using \textsc{Fortran} Particle-In-Cell simulations with Herbert Gunell and Mark Koepke 
%\item I demonstrated that stationary inertial Alfv\'{e}n wave structure observed in UCLA's large plasma device was not a measurement artifact
%\item I procured, installed, and tested custom interlock and mass-flow control system for laboratory plasma system
\end{itemize}
}
%%##!!!!!~~~~~
%% I have commented the above for easy copy and paste when filling out a form
%Graduate Research Assistant for Mark Koepke, West Virginia University,
%During my Ph.D. education, I was responsible for development and analysis of  simulation codes and analytical models for dust grains with time-dependent charge in plasma. 

%I Demonstrated how magnetized-orbit dust grain transport depends on charging rate for abrupt and gradual inhomogeneities.

%I Predicted grain trajectories for the Auburn Magnetized Dusty Plasma experiment. 

%With Michael Zimmerman I developed a symplectic, single particle simulation code to study dust grain transport due to non-stationary charging efects,  including an innovative iterative leapfrog scheme to treat non-linear drag on dust grains.


%I Developed semi-analytical model for dust grain transport in an abrupt plasma  inhomogeneity.


%% for cwi or uva cv, is this really necessary?
\cventry{August 2005 -- May 2015}{Teaching Assitant for Greg Puskar}{West Virginia University}{Morgantown, WV, USA}{}{Responsible for establishing, teaching, and grading undergraduate physics laboratory courses for 100 students each semester
%, and I am happy to work one-on-one to clarify technical and conceptual details as needed. 
%while fostering a team environment
%I enjoy balancing lecturing, experimentation, evaluating, and tutoring demands. 
%% should I make an itemized list here?
\begin{itemize}
\item Provided constructive feedback on assignments 
\item Helped my students develop critical thinking skills and problem solving techniques
\item Illustrated how physics concepts and equations apply to their world through relatable examples
\end{itemize}
}

%% entry for radioshack job?
%------------------------------------------------

%% ??
%\subsection{Miscellaneous}

% Radioshack job can go here I suppose.
%\cventry{October 2004 -- August 2005}{Sales Associate}{RadioShack}{Buffalo, NY}{}{As a sales associate during my undergraduate career, I was responsible for greeting and serving customers with enthusiasm and energy. I was also responsible for maintaining technical knowledge in order to confidently offer customers cutting-edge solutions. My other responsibilities also included maintaining the store’s policies and procedures regarding stocking, cleaning, and merchandising.}

%----------------------------------------------------------------------------------------
%	AWARDS SECTION
%----------------------------------------------------------------------------------------

% I have no awards :`(
%\section{Awards}

%\cvitem{2011}{}


%----------------------------------------------------------------------------------------
%	COMMUNICATION SKILLS SECTION, or alternatively use this as a section to discuss papers and conferences
%----------------------------------------------------------------------------------------

%----------------------------------------------------------------------------------------
%	PUBLICATIONS
%----------------------------------------------------------------------------------------
%% for industry jobs, might consider making a separate pdf for the publications.
%\section{Publications}

%\cvitem{2017}{J. J. Walker, J. S. Halekas, M. Hor\'{a}nyi, J. R. Szalay, and A. R. Poppe, Evidence for Detection of Energetic Neutral Atoms by \textsc{Ladee}, \textit{Planetary and Space Science}, \textbf{139}, pp 31-36 (2017).\newline \url{http://dx.doi.org/10.1016/j.pss.2017.03.002}}
%
%\cvitem{2016}{J. J. Walker, M.E. Koepke, and M. I. Zimmerman, Predictions for Gyro-phase Drift in MDPX, \textit{Physics of Plasmas} \textbf{23}, 103707, (2016).\newline \url{http://dx.doi.org/10.1063/1.4966202}}
%
%\cvitem{2014}{J. J. Walker, M.E. Koepke, M. I. Zimmerman, W. M. Farrell, and V. I. Demidov, Analytical model for gyro-phase drift arising from abrupt inhomogeneity, \textit{Journal of Plasma Physics} \textbf{80}, 3, pp 395-404, (2014).\newline  \url{https://doi.org/10.1017/S0022377813001359} } 
%
%\cvitem{2013}{M.E. Koepke, J. J. Walker, M. I. Zimmerman, W. M. Farrell, and V. I. Demidov, Signature of Gyro-phase Drift, \textit{Journal of Plasma Physics} \textbf{79}, 6, pp 1099-1105, (2013).\newline \url{https://doi.org/10.1017/S0022377813001128}}
%
%\cvitem{2009}{H. Gunell, J. J. Walker, M. E. Koepke, T. Hurtig, N. Brenning, and H. Nilsson, Numerical experiments on plasmoids entering a transverse magnetic field, \textit{Physics of Plasmas} \textbf{16}, 112901, (2009).\newline \url{http://dx.doi.org/10.1063/1.3267860}}


%% comment out the conferences for the generic resume
%----------------------------------------------------------------------------------------
%% put conferences here too??
%\section{Conference and Workshop Contributed Posters}

%\cvitem{2016}{J. J. Walker, J. S. Halekas, M. Hor\'{a}nyi, J. R. Szalay, A. R. Poppe, and M. C. Lue, ``\textit{ENA Measurements of the Lunar Surface using LDEX}", Meeting of the American Geophysical Union, San Francisco, CA, 12--16 Dec} 

%\cvitem{2015}{J. J. Walker, J. S. Halekas, M. Hor\'{a}nyi, J. R. Szalay, and A. R. Poppe, ``\textit{Measurement of Energetic Neutral Atom Flux in the Lunar Exosphere using the LDEX Instrument}", Meeting of the American Geophysical Union, San Francisco, CA, 14--18 Dec} 

%\cvitem{2014}{J. J. Walker, M. E. Koepke, M. I. Zimmerman, W. M. Farrell, and V. I. Demidov, ``\textit{Non-Stationary Charging Dynamics of an Inhomogeneous Granule-Plasma Multi-phase System}", Department of Energy Center for Predictive Control of Plasma Kinetics: Multi-Phase and Bounded Systems, College Park, MD, 15--16 May}

%\cvitem{2013}{M. Koepke, J. Tucker, C. Freeman, D. Meisel, J. Walker, M. Zimmerman, W. Farrell, V. Demidov, ``\textit{Laboratory analysis of granular materials properties, size distributions, chemical and mineralogical compositions relevant to dust-grain charging investigations}", The Fourth Moscow Solar System Symposium (4M-S3), IKI RAS (Space Research Institute, Russian Academy of Sciences), Moscow, Russia, 14--18 October 2013} 

%\cvitem{2013}{J. J. Walker, M.E. Koepke, M. I. Zimmerman, W. M. Farrell, and V. I. Demidov, ``\textit{Gyrophase Drift in Laboratory and Industrial Regimes}", Department of Energy Center for Predictive Control of Plasma Kinetics: Multi-Phase and Bounded Systems, College Park, MD, 16--17 May}

%% Not sure if I should include this or not; I made the poster but Mark presented it.
%\cvitem{2013}{J. J. Walker, M.E. Koepke, M. I. Zimmerman, W. M. Farrell, and V. I. Demidov, ``\textit{Signature of Gyro-phase Drift}", Meeting of the European Physical Societies Division of Plasma Physics, Espoo, Finland, 1--5 July}

%\cvitem{2012}{J. J. Walker, M.E. Koepke, M. I. Zimmerman, W. M. Farrell, and V. I. Demidov, ``\textit{Magnitude and Direction of Fine-Particle Gyrophase Drift}", Meeting of the American Physical Society's Division of Plasma Physics, Providence, RI, 29 October--2 November}

%\cvitem{2012}{J. J. Walker, M.E. Koepke, M. I. Zimmerman, W. M. Farrell, and V. I. Demidov, ``\textit{Magnitude and Direction of Gyrophase Drift in Dusty Plasmas with Structured Inhomogeneity}", Department of Energy Center for Predictive Control of Plasma Kinetics: Multi-Phase and Bounded Systems, Princeton Plasma Physics Laboratory, Princeton, NJ, 17--18 May}

%\cvitem{2012}{M. I. Zimmerman, M.E. Koepke,  J. J. Walker, and W. M. Farrell, ``\textit{A Numerical Investigation of Dust Gyrophase Drift}", 13th Workshop on the Physics of Dusty Plasmas organized by Baylor University and the Center for Astrophysics, Space Physics and Engineering Research (CASPER), Baylor University, Waco, TX, 20--23 May}

%----------------------------------------------------------------------------------------
%	Relevant Coursework
%----------------------------------------------------------------------------------------
%% figure out how to fix this to make it look less terrible
%\section{Selected Coursework}
%\section{Relevant Courses}
%%\section{Graduate Courses Within Speciality}
%\cvitem{--}{Machine Learning by Stanford University on Coursera earned on August 25, 2017\newline \url{https://lnkd.in/gt743Rs}}
%\cvitem{--}{Stellar Structures}
%\cvitem{--}{Numerical Simulations of Partial Differential equations}{}
%\cvitem{--}{Computer Simulation of Plasma}{}
%\cvitem{--}{Advanced Magnetohydrodynamics Theory of Plasma}{}
%\cvitem{--}{Non-Linear Dynamics}{}
%\cvitem{--}{Advanced Kinetic Theory of Plasma}{}

%\cvitem{--}{Essentials of Leadership}{}

% probably don't need outreach section in a generic resume.
%----------------------------------------------------------------------------------------
%	OUTREACH SECTION
%----------------------------------------------------------------------------------------

%\section{Outreach}

%\cvitem{2014}{Science outreach activities representing the WVU Physics and Astronomy Department for potential students, grad students, and donors at West Virginia State Fair}{}


%----------------------------------------------------------------------------------------
%	LANGUAGES SECTION
%----------------------------------------------------------------------------------------

%\section{Languages}

%\cvitemwithcomment{English}{Native}{}
%\cvitemwithcomment{Spanish}{Basic}{Basic words and phrases only}
%\cvitemwithcomment{Swedish}{Intermediate}{}
%\cvitemwithcomment{Ukrainian}{Basic}{}
%\cvitemwithcomment{German}{Intermediate}{}
%\cvitemwithcomment{Polish}{Basic}{}
%\cvitemwithcomment{Nederlands}{Basic}



%----------------------------------------------------------------------------------------
%	Collaborators external to WVU
%----------------------------------------------------------------------------------------
%% Do I need this section??

%----------------------------------------------------------------------------------------
%	Professional Affiliations
%----------------------------------------------------------------------------------------

%----------------------------------------------------------------------------------------
%	Professional Service
%----------------------------------------------------------------------------------------
%----------------------------------------------------------------------------------------
%	CERTIFICATES?
%----------------------------------------------------------------------------------------

%\section{Certificates}

%\cvitem{2017}{Machine Learning by Stanford University on Coursera earned on August 25, 2017\newline \url{https://lnkd.in/gt743Rs}}

%----------------------------------------------------------------------------------------
%	SKILLS SECTION, should I make an experimental skills/hardware skills section??
%----------------------------------------------------------------------------------------

%% Need: experimental skills: vacuum design and plasma experiments...
%% don't forget: spice kernels!
%% vector plots using tikz

\section{Skills}
%\section{Skillset}

%% I will leave the old way I broke this up into advanced/intermediate/basic, and instead just put all the skill together and leave out any skills I think that I do not really have.

%\cvitem{Advanced}{Statistical Analysis, \textsc{Matlab}, \textsc{Octave}, \LaTeX, Modelling and Simulation, Finite-Differencing Techniques, Particle-In-Cell codes}
%\cvitem{Intermediate}{Machine Learning, \textsc{Python}, \textsc{NumPy}, \textsc{matplotlib}, \textsc{scikit-learn}, \textsc{pandas}, \textsc{IDL}, \textsc{Bash} and Shell Programming, \textsc{C$^{\mathrm{++}}$},  \textsc{Fortran}, Linux, Mac OSX, Microsoft Windows, Finite Element Analysis, OpenOffice, Microsoft Office, Procurement,  Vacuum Hardware, Experimental Apparatus Design}
%\cvitem{Basic}{\textsc{Java},  \textsc{Html}, \textsc{Sql}}

%% I believe this was the skillset I used for the cwi postdoc, maybe I do not need all of these for uva postdoc
%\cvitem{}{Computer Simulation of Plasmas, Machine Learning, Statistical Analysis, \textsc{Matlab}, \textsc{Octave}, \LaTeX, Finite-Differencing Techniques, Finite Element Analysis, Spice Kernels, Particle-In-Cell codes, \textsc{Python}, \textsc{NumPy}, \textsc{matplotlib}, \textsc{IDL}, \textsc{Bash} and Shell Programming, \textsc{C$^{\mathrm{++}}$},  \textsc{Fortran}, Linux, Mac OSX, Microsoft Windows, OpenOffice, Microsoft Office}

% for an industry job, no one is going to know what spice kernels are. Spice Kernels
\cvitem{}{Numerical Methods for Differential Equations, \textsc{Matlab}, \textsc{Octave}, \LaTeX, Finite-Differencing Techniques, Particle-In-Cell codes, Machine Learning, Statistical Analysis,  \textsc{SQL}, \textsc{Python}, \textsc{Bash} and Shell Programming, C, \textsc{C$^{\mathrm{++}}$},  \textsc{Fortran}, Linux, Mac OSX, Microsoft Windows, OpenOffice, Microsoft Office, \textsc{IDL}}

%% I've listed these skills in a format that is easy for copying and pasting into job application websites
%Physics Education, Numerical Methods, Matlab, Octave, LaTeX, Finite-Differencing Techniques, Particle-In-Cell codes, Machine Learning, Statistical Analysis, SQL, Python, NumPy, matplotlib, IDL, Bash and Shell Programming, C, C++,  Fortran, Linux, Mac OSX, Microsoft Windows, OpenOffice, Microsoft Office

%----------------------------------------------------------------------------------------
%	INTERESTS SECTION
%----------------------------------------------------------------------------------------

\section{Hobbies and Interests}

\renewcommand{\listitemsymbol}{-~} % Changes the symbol used for lists

\cvlistdoubleitem{Guitar}{Cello}
\cvlistdoubleitem{Chess}{Boardgames}
\cvlistdoubleitem{Celestial Navigation}{Geography}
\cvlistdoubleitem{Hamilton-Jacobi Mechanics}{Economics}
%\cvlistdoubleitem{Drawing}{Formula One}
%\cvlistdoubleitem{Weightlifting}{Formula One}
\cvlistdoubleitem{Linguistics}{History}


%% Probably want to NOT include:
%\cvlistdoubleitem{Model Railroading}{Hamilton-Jacobi Mechanics}

%% use this command to list just one?
%\cvlistitem

%----------------------------------------------------------------------------------------
%	Attach additional pdf documents here!
%----------------------------------------------------------------------------------------

% use this command to include a pdf
% the pages={1-} option ensures that the entire pdf document will be included
%\includepdf[pages={1-}]{grade_list.pdf}

%----------------------------------------------------------------------------------------

\end{document}
